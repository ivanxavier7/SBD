\documentclass[11pt,a4paper,twoside]{article}
\usepackage[portuguese]{babel}
\usepackage{graphicx}  		%% display images
\usepackage{float}			%% make graphics float in place
\usepackage{indentfirst}
\usepackage{hyphenat}
\usepackage{xcolor}
\usepackage{subcaption}		%% allows for subgigures
\usepackage{listings} 		%% for listing code
\usepackage[top=2.50cm, bottom=2.50cm,left=2.50cm, right=2.50cm]{geometry} % margens
\usepackage[colorlinks=true,linkcolor=black,urlcolor=black,bookmarksopen=true]{hyperref} %% Make hyperlinks in index
\usepackage{bookmark} 		% Bookmarks for pdf file
\renewcommand{\baselinestretch}{1.2} 
\usepackage{fancyhdr} 		% creates fancy footers and headers
\pagestyle{fancy}
\lfoot{{\footnotesize Tecnologias de Informação}}
\rfoot{{\footnotesize ESTGA}}
\rhead{Projecto Temático em Desenvolvimento de Aplicações}
\lhead{}
\usepackage{listings}

\definecolor{codegreen}{rgb}{0,0.6,0}
\definecolor{codegray}{rgb}{0.5,0.5,0.5}
\definecolor{codepurple}{rgb}{0.58,0,0.82}
\definecolor{backcolour}{rgb}{0.95,0.95,0.92}

\lstdefinestyle{mystyle}{
	backgroundcolor=\color{backcolour},   
	commentstyle=\color{codegreen},
	keywordstyle=\color{magenta},
	numberstyle=\tiny\color{codegray},
	stringstyle=\color{codepurple},
	basicstyle=\ttfamily\footnotesize,
	breakatwhitespace=false,         
	breaklines=true,                 
	captionpos=b,                    
	keepspaces=true,                 
	numbers=left,                    
	numbersep=5pt,                  
	showspaces=false,                
	showstringspaces=false,
	showtabs=false,                  
	tabsize=2
}

\lstset{style=mystyle}

% Title Page
\title{}
\author{}


\begin{document}

\begin{titlepage}
	\centering
	\vfill
	\includegraphics[width=7cm]{image/ESTGA} % also works with logo.pdf
	\vfill
	
	{\bfseries\Large
		Projeto temático em Desenvolvimento de Aplicações\\
		PyClinic - Guia Instalação \\
		\vskip2cm
		%A. Uthor\\
	}    
	
	\vfill
	\textbf{Grupo 5}
	
	António Bento (97737)
	
	Diogo Matos (98017)
	
	Ivan Xavier (92441)
	
	Maykol Santos (74079)
	
	Ricardo Fernandes (49880)
	
	Simão Julião (98045)
	\vfill
\end{titlepage}


\section{Introdução}
\pagenumbering{arabic}
O presente documento explica sucintamente como instalar e iniciar a aplicação PyClinic, tanto em Windows (Secção 4) como em Linux (Secção 5).
\section{Hardware}
Em termos de hardware, as especificações mínimas do PyClinic são as seguintes:
\begin{itemize}
\item 4GB de RAM;
\item Processador x86 criado a partir de 2010;
\end{itemize}

\section{Software}
Para a utilização da aplicação PyClinic, os requisitos de software são os seguintes:
\begin{itemize}
	\item Sistema Operativo de 64 bits;
	\item Python 3.8 para Linux ou Python 3.9 em Windows 10;
\end{itemize}


\section{Windows 10}

Para utilizar a aplicação PyClinic em Windows 10, apenas necessita de aceder ao ficheiro (.zip), extraindo toda a sua informação para o seu computador. Após a sua extração, necessita apenas de iniciar o ficheiro PyClinic.exe.

Para facilitar o uso no seu computador, pode criar um atalho apenas do executável para o seu ambiente de trabalho. A partir de agora poderá utilizar o PyClinic!

\section{Linux}

Para utilizar a aplicação PyClinic em Linux, necessita de aceder ao ficheiro (.tar.xz), extraindo toda a sua informação para o seu computador. Após a extração, necessitará de utilizar a consola do Linux, na pasta extraída e utilizar o comando:
%\usepackage{listings}
\begin{lstlisting}[language=bash]
	$ chmod +x PyClinic.sh
\end{lstlisting}
podendo de seguida fazer duplo clique na aplicação para a iniciar. A partir de agora poderá utilizar o PyClinic!


\end{document}          



